\section{Sequences in metric spaces}

Like in the real numbers, we can define sequences in all metric spaces. And in fact, we need to understand sequences in real numbers to define sequences in metric spaces. 

\begin{definition}[Sequence]\label{def:sequence}
    For a set $X$, we call a sequence in $X$ a map $x: \mathbb{N} \to X$ and write $x_{n} = x(n)$. For the entire sequence, we write $(x_{n})_{n}^{\infty} \subset X$.
\end{definition}

\begin{definition}[Convergence and limit]\label{def:conv_lim}
    Let $(x_{n})_{n}^{\infty}$ be a sequence in a metric space $(X,d)$. We say $(x_{n})_{n}^{\infty}$ converges to $x \in X$ if $d(x_{n}, x) \to 0$ as a sequence of real numbers.

    Equivalently, $\forall \epsilon > 0 \;  \exists N \in \mathbb{N} \text{ s.t. } \forall n \geq N: d(x_{n}, x) < \epsilon$.

    We write $\lim_{n \to \infty} x_{n} = x$, or $x_{n} \to x$. 
\end{definition}


\begin{lemma}[Limit is unique]\label{lem:lim_unique}
    Let $(X, d)$ be a metric space and $(x_{n})_{n}^{\infty} \subset X$ a sequence with $x_{n} \to x, x_{n} \to y$. Then $x = y$. 
\end{lemma}
\begin{proof}
    Trivial.
\end{proof}

\begin{definition}[Subsequence]\label{def:subsequence}
    Let $(x_{n})_{n}^{\infty}$ be a sequence in $X$. We define a subsequence as any sequence of the form $\left(x_{f(k)}\right)_{k}^{\infty}$ where $f: \mathbb{N} \to \mathbb{N}$ is increasing. We write $f(k) = n_{k}$, so the subsequence is denoted $\left(x_{n_{k}}\right)_{k}^{\infty}$.
\end{definition}

\begin{definition}[Accumulation point]\label{def:acc_point}
    Let $(X,d)$ be a metric space.

    (1) For a subset $Y \subset X$, we say that $y \in X$ is an accumulation point of $Y$ if $\exists (y_{n})_{n}^{\infty} \subset Y$ such that $y_{n} \to y$.

    (2) For a sequence $(x_{n})_{n}^{\infty} \subset X$, we say that $x \in X$ is an accumulation point if $\exists$ a subsequence $x_{n_{k}} \to x$. 
\end{definition}

\begin{lemma}[Subsequences and limit]\label{lem:subseq_lim}
    A sequence $(x_{n})_{n} \subset X$, for $(X,d)$ metric space, converges to $x$ if and only if every subsequence $(x_{n_{k}})_{k}^{\infty}$ converges to $x$.
\end{lemma}
\begin{proof}
    $\impliedby$: If every subsequence converges to $x$, since $(x_{n})_{n}^{\infty}$ is also a subsequence, the limit is $x$.

    $\implies$: Let $(x_{n})_{n}^{\infty} \to x$ and $(x_{n_{k}})_{k}^{\infty}$ any subsequence. For $\epsilon > 0$ find $N \in \mathbb{N}$ such that $d(x_{n},x) < \epsilon$ for $n \geq N$, then since $n_{k} \geq k$, we get $d(x_{n_{k}}, 0) < \epsilon$, so $x_{n_{k}} \to x$ as well.
\end{proof}

\begin{lemma}[Subsequence and limit 2]\label{lem:subseq_lim_2}
    Under the same assumptions, $x_{n} \to x$ if and only if every subsequence $(x_{n_{k}})_{k}^{\infty}$ has a sub-subsequence $x_{n_{k}} \to x$.
\end{lemma}
\begin{proof}
    Exercise.
\end{proof}

\begin{definition}[Cauchy sequence]\label{def:cauchy_seq}
    In a metric space, $(x_{n})_{n}^{\infty}$ is Cauchy if 
    \begin{align*}
        \forall \epsilon > 0 \; \exists N \in \mathbb{N} \text{ s.t. } \forall n,m \geq N: d(x_{n}, x_{m}) < \epsilon.
    \end{align*}
\end{definition}

\begin{definition}[Completeness]\label{def:completeness}
   $(X,d)$ is complete if every Cauchy sequence in it converges.
\end{definition}

\begin{eg} 
    Consider $(\mathbb{R}, d_{\text{Euclidean}})$. The space $A = (0,\infty)$ is \textit{not} complete, but $A = [a,b]$ is. $\mathbb{Q} \subset \mathbb{R}$ is not complete, because you can have a sequence going to $\sqrt{2}$, which is not in $\mathbb{Q}$.
\end{eg}

\begin{remark}
    There is a conflict of notation in the following proofs, because for a sequence $(x_{n})_{n}^{\infty} \subset \mathbb{R}^{n}$, $x_{n}$ could either mean $x(n)$ or the $n$-th component of the vector $x$. So in the following two proofs, $m$ will be the index in the sequence and $i,j$ the coordinate index. And then $x_{m,i}$ means the $i$-th component of $x(m)$.
\end{remark}

\begin{lemma}[Coordinate wise convergence ]\label{lem:coordin_wise_conv}
    Let $(x_{m})_{m}^{\infty}$ be a sequence. Then $x_{m} \to x$ if and only if $x_{m,i} \to x_{i}$ for every $1 \leq i \leq n$.
\end{lemma}
\begin{proof}
    $\implies$: Assume $x_{m} \to x$. This means for all $\epsilon > 0$ there exists $N \in \mathbb{N}$ such that $\lVert x_{m} - x \rVert < \epsilon$ if $m \geq N$. Given $1 \leq i \leq n$, notice
    \begin{align*}
        \left| x_{m,i} - x_{i} \right| \leq \sqrt{ \sum_{j=1}^{n} \left| x_{m,j} - x_{j} \right|^{2} } = \lVert x_{m} - x \rVert < \epsilon.
    \end{align*}
    This way, $x_{m,i} \to x_{i}$.

    $\impliedby$: Let $\epsilon > 0$. By assumption for all $1 \leq i \leq n$, $\exists N_{i} \in \mathbb{N} \text{ s.t. } \forall m \geq N_{i}: \left| x_{m,i} - x_{i} \right| < \frac{\epsilon}{\sqrt{n}}$. If we define $N = \max_{1 \leq i \leq n} N_{i}$, then for $m \geq N$,
    \begin{align*}
        \lVert x_{m} - x \rVert  = \sqrt{\sum_{j=1}^{n} \left| x_{m,i} - x_{i} \right|^{2}  } \leq \sqrt{n \frac{\epsilon^{2}}{n}} = \epsilon.
    \end{align*}
    This way, $x_{m} \to x$ as a whole.
\end{proof}

\begin{theorem}[Euclidean $\mathbb{R}^{n}$ is complete]\label{thm:reals_complete}
    The metric space $(\mathbb{R}^{n}, d_{\text{Euclidean}})$ with the Euclidean metric \ref{def:euclid_norm} is complete.
\end{theorem}
\begin{proof}
    So let $(x_{n})_{n}^{\infty}$ be a Cauchy sequence in $\mathbb{R}^{n}$. Because $\left| x_{i} - y_{i} \right| \leq \lVert x - y \rVert  $ for $x,y \in \mathbb{R}^{n}$, we get that all the component-wise sequence are Cauchy. But those are in $\mathbb{R}$, so they converge. By \ref{lem:coordin_wise_conv}, $x_{n} \to x$.
\end{proof}

