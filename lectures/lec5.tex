\begin{corollary}\label{cor:compactness}
	The last proposition gives us the following corollaries:
	\begin{enumerate}
		\item $K \subset X$ compact $\implies K$ closed.
		\item $K \subset X$ compact $\implies K$ complete.
		\item $K \subset X$, $A \subset X$ closed $\implies$ $K \cap A$ compact.
	\end{enumerate}
\end{corollary}
\begin{proof}
	(1) is simple, we can use sequential closedness \ref{lem:seq_open_closed}. If a sequence converges, then it must converge to the limit of any of its subsequences. (2) also follows since if a Cauchy sequence has a convergent subsequence, then it also converges as a whole, and converges into $K$, because it is closed by (1). (3) also follows from sequential compactness and the fact that the limit of the subsequence must be in $A$.
\end{proof}

\begin{prop}[Continuous image of a compact set is compact]\label{prop:contin_image_of_compact}
	Let $(X, d_{X})$ and $(Y, d_{Y})$ be metric spaces and $K \subset X$ be compact. If $f: X \to Y$ is continuous, then $f(K)$ is compact in $Y$.
\end{prop}
\begin{proof}
	The goal is to show that $f(K)$ is topologically compact, so for $\mathcal{V}$ an open cover of $f(K)$, I want to find a finite subcover. So let $\mathcal{U} = \left\{ f^{-1}(V) \mid V \in \mathcal{V} \right\}$, which is an open (by continuity) cover of $K$. Extract a finite subcover of $K$ and map it back to the codomain.
\end{proof}

\begin{theorem}[Extreme value theorem]\label{thm:extreme_val_theorem}
	Let $(X,d)$ be a metric space and $f:K \to \mathbb{R}$ be continuous and $K \subset X$ compact. Then both $\sup\left\{ f(x) \mid x \in K \right\}$ and $\inf\left\{ f(x) \mid x \in K \right\}$ are attained.
\end{theorem}
\begin{proof}
	We will prove it for the supremum only. By definition of the supremum $s = \sup_{x \in K} \left\{ f(x) \right\}$, there exists a sequence $(x_{n})_{n}^{\infty} \subset K$ such that $f(x_{n}) \to s$. By compactness, this has a convergent subsequence $x_{n_{k}} \to \bar{x}$. By sequential continuity, $f(\bar{x}) = s$.
\end{proof}

\begin{remark}
	We take a subset $K \subset \mathbb{R}^{n}$ to be bounded w.r.t. the Euclidean distance if there exists $N \in \mathbb{N}$ such that $\lVert x \rVert < N$ for all $x \in K$, equivalently $K \subset B_{N}(0)$.
\end{remark}

\begin{theorem}[Heine-Borel]\label{thm:heine-borel}
	For the metric space $\mathbb{R}^{n}$ with the Euclidean distance, $K \subset \mathbb{R}^{n}$ is compact if and only if it is bounded and closed.
\end{theorem}
\begin{proof}
	$\implies$: $K$ is closed by Corollary \ref{cor:compactness}. If $K$ is unbounded, $\exists (x_{n})_{n}^{\infty} \subset K$ such that $\lVert x_{n} \rVert \geq N$ for all $N \in \mathbb{N}$. Any subsequence will hence diverge. Indeed, $x_{n_{k}} \to x$ would imply $\lVert x_{n_{k}}-x \rVert \to 0$, and by triangle inequality, \ref{lem:triangle_ineq}
	\begin{align*}
		\lVert x_{n_{k}} \rVert  \leq \lVert x \rVert + \lVert x_{n_{k}} - x \rVert,
	\end{align*}
	a contradiction to $x_{n_{k}}$ being arbitrarily large.

	$\impliedby$: It suffices to show that for $N \in \mathbb{N}$, $[-N, N]^{n} \subset \mathbb{R}^{n}$ is compact, assuming that $K \subset \mathbb{R}^{n}$ is closed and bounded by $B_{N}(0)$. This is because $K \subset B_{N}(0) \subset [-N, N]^{n}$, and a closed subset of a compact set is compact, so $K$ would be compact, by Corollary \ref{cor:compactness}.

	The proof will have to use the Bolzano-Weierstrass Theorem in $\mathbb{R}$. Given a sequence $(x_{k})_{k}^{\infty}$, we can write this component-wise
	\begin{align*}
		x_{k} = (x_{k,1}, \dots, x_{k, n}).
	\end{align*}

	Consider the sequences $(x_{k,i})_{k=1}^{\infty} \subset [-N,N]$ for $1 \le i \le n$.
	By Bolzano--Weierstrass, there exists a strictly increasing sequence of indices
	$(k_m^{(1)})_{m=1}^{\infty}$ such that
	\begin{align*}
		x_{k_m^{(1)},1} \to \ell_1 \in [-N,N]
		\quad \text{as } m \to \infty.
	\end{align*}

	Now consider the sequence $(x_{k_m^{(1)},2})_{m=1}^{\infty} \subset [-N,N]$. Again by Bolzano--Weierstrass, there exists a strictly increasing sequence $(k_m^{(2)})_{m=1}^{\infty}$ which is a subsequence of $(k_m^{(1)})_{m=1}^{\infty}$ such that
	\begin{align*}
		x_{k_m^{(2)},2} \to \ell_2 \in [-N,N] \quad \text{as } m \to \infty.
	\end{align*}
	Since $(k_m^{(2)})$ is a subsequence of $(k_m^{(1)})$, we still have
	\begin{align*}
		x_{k_m^{(2)},1} \to \ell_1.
	\end{align*}

	Proceeding inductively, for each $j=1,\dots,n$ we obtain a strictly increasing sequence $(k_m^{(j)})_{m=1}^{\infty}$ which is a subsequence of $(k_m^{(j-1)})_{m=1}^{\infty}$ and such that
	\begin{align*}
		x_{k_m^{(j)},j} \to \ell_j \in [-N,N] \quad \text{as } m \to \infty,
	\end{align*}
	and for all $i \le j$,
	\begin{align*}
		x_{k_m^{(j)},i} \to \ell_i.
	\end{align*}

	After $n$ steps we obtain $(k_m^{(n)})_{m=1}^{\infty}$ such that
	\begin{align*}
		x_{k_m^{(n)},i} \to \ell_i \quad \text{for all } i=1,\dots,n.
	\end{align*}

	Define $x := (\ell_1,\dots,\ell_n) \in [-N,N]^n$. Then $x_{k_m^{(n)}} \to x$ componentwise, hence $x_{k_m^{(n)}} \to x$ in $\mathbb{R}^n$ by Lemma \ref{lem:coordin_wise_conv}.
\end{proof}

\begin{eg}
	Consider $X = \mathbb{R}$ and $A = (0,1]$. Why is this not compact? Because
	\begin{align*}
		A \subset \bigcup_{n = 1}^{\infty} \left( \frac{1}{n}, 2 \right)
	\end{align*}
	gives us a cover of $A$. But any finite cover of this will definitely miss some numbers of $A$.
\end{eg}

\begin{prop}[Uniform continuity]\label{prop:unif_cont}
	Let $f: X \to Y$ be a continuous map between metric spaces. If $K \subset X$ is compact, then $f|_{K}$ is uniformly continuous.
\end{prop}
\begin{proof}
	The goal is to show that given $\epsilon > 0$, there exists $\delta > 0$ such that $\forall x \in K: f(B_{\delta}(x)) \subset B_{\epsilon}(f(x))$. Using continuity, $\forall x \in K$, there exists $\delta(x) > 0$ such that $f(B_{\delta(x)}(x)) \subset B_{\frac{\epsilon}{2}}(f(x))$. Define
	\begin{align*}
		\mathcal{U} = \left\{ B_{\frac{\delta(x)}{2}}(x) \mid x \in K \right\}.
	\end{align*}
	This is an open cover of $K$. So it has a finite subcover. So 
	\begin{align*}
		K \subset \bigcup_{i=1}^{n} B_{\frac{\delta(x_{i})}{2}}(x_{i})
	\end{align*}
	and we define $\delta = \min_{1 \leq i \leq n}\left\{ \frac{\delta(x_{i})}{2} \right\}$. 

	If $x \in K$ and $y \in B_{\delta}(x)$, then $x \in B_{\delta(x_{i}) / 2}(x_{i})$ for some $i$. Consider that 
	\begin{align*}
		d(x,y) < \delta \leq \frac{\delta(x_{i})}{2} , \quad d(x_{i}, y) < \frac{\delta(x_{i})}{2} + \frac{\delta(x_{i})}{2} < \delta(x_{i}).
	\end{align*}
	That means
	\begin{align*}
		f(B_{\delta}(x)) \subset f(B_{\delta(x_{i})}(x_{i})) \subset B_{\epsilon / 2} (f(x_{i})) \subset B_{\epsilon}(f(x)). 
	\end{align*}
	
\end{proof}

\section{Connectedness}

In any metric space, $X, \emptyset$ are always both open and closed (clopen). Are there other clopen sets and how many? Or, consider the following question. For the sphere $X = \mathbb{S}^{2} = \left\{ v \in \mathbb{R}^{3} \mid \lVert v \rVert = 1 \right\}$ and the Euclidean distance, is it possible to write $\mathbb{S} = U \cup V$ with $U \cap V = \emptyset$ such that $U, V$ are both open and non-empty? Intuitively, it seems quite improbable. One can ask the same question for $(0,1) \subset \mathbb{R}$. We want to explore the concept of connected space in this section and end up answering these seemingly unrelated questions.

\begin{definition}[Connectedness]\label{def:connectedness}
	Let $(X,d)$ be a metric space and $A \subset X$ a nonempty subset.

	$X$ is \textit{disconnected} if $\exists U, V \subset A$ open, disjoint and nonempty such that
	\begin{align*}
		A \subset U \cup V \text{ and } A \cap  U \neq \emptyset, A \cap V \neq \emptyset.
	\end{align*}

	Any subset $A$ that is not disconnected is called \textit{connected}.
\end{definition}

\begin{eg}
	In $\mathbb{R}$, the set $A = (0,1) \cup (2,3)$ is immediately seen to be disconnected, as is $A = \left\{ 0,1 \right\}$.
\end{eg}

